\section{Motivación}
Hoy en día, es mucha la gente que se pasa buena parte del día frente a un ordenador, tanto por trabajo como en su tiempo libre. Esto, por supuesto, puede suponer problemas para la salud si no se toman las precauciones necesarias. Por ejemplo, problemas de vista por pasar excesivas horas mirando un monitor, aunque en este frente ya hay una buena cantidad de divulgación e investigación.

Sin embargo, el periférico que más usamos es el teclado y, sin embargo, su \emph{problemático} diseño apenas ha cambiado desde que existen los ordenadores y puede resultar en diversas lesiones y enfermedades en las muñecas.

\section{Objetivo}
El principal fin de este trabajo va a ser el diseño de un teclado que se adapte mejor a la anatomía humana y que, a su vez, incorpore mejoras que lo hagan accesible a gente con diversas discapacidades:
\begin{itemize}
  \item Teclas con letras grandes y alto contraste
  \item Un joystick integrado que permita controlar el teclado
  \item Solenoide para mayor feedback táctil y sonoro
\end{itemize}
