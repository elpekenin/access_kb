\subsection{Objetivos}
Las metas principales a la hora de diseñar el teclado han sido el usar la menor cantidad de pines posible para las tareas ``básicas'' y el exponer todos los pines restantes así como varias tomas de alimentación. \newline
De esta forma, el teclado puede servir como placa de pruebas donde desarrollar drivers para otro hardware y además es modular, ya que nos permite añadir más periféricos (por ejemplo, para sonido) en el futuro sin necesidad de tener que fabricar una nueva PCB.

\subsection{Distribución}
He optado por una disposición ortolineal, split y (por ahora) QWERTY, reduciendo un par de columnas respecto al tamaño habitual ya que algunas de esas teclas raramente se usan, y al tener una forma simétrica es más sencillo de diseñar. 

\image{layout}{\textwidth}{Diseño aproximado del teclado}
