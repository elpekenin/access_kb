\subsection{Instalación y configuración de QMK}
Para usar QMK\cite{qmk} instalamos su CLI ejecutando \cli{pip install qmk} ya que se trata de una librería escrita en Python y está disponible en el repositorio de paquetes de Python(pip). \newline
Tras esto, descargamos el código fuente con \cli{qmk setup}, a este comando le podemos pasar como parámetro nuestro fork del repositorio (en mi caso \cli{qmk setup elpekenin/qmk_firmware}), para poder usar git, ya que no tendremos permisos en el repositorio oficial. Este comando también nos instalará los compiladores necesarios y comprobará las udev de nuestro sistema Linux, para que podamos trabajar con los dipositivos sin problema, en caso de que no estén bien configuradas podemos usar copiar el archivo \icode{50-qmk.rules} en \icode{/etc/udev/rules.d/}. Finalmente podemos ejecutar \cli{qmk doctor} para comprobar el estado de QMK. \newline
Para poder hacer debug con \cli{qmk console} he necesitado añadir una línea a dicho archivo: \newline
\icode{KERNEL==``hidraw'', SUBSYSTEM==``hidraw'', MODE=``0666'', TAG+=``uaccess'', TAG+=``udev-acl''} \newline

\hr
Opcionalmente, podríamos instalar LVGL\cite{lvgl} funciones gráficas más complejas en la pantalla LCD, en vez de usar el driver de QMK. Esto se hace con \newline
\cli{git submodule add -b release/v8.2 https://github.com/lvgl/lvgl.git lib/lvgl} (desde el directorio base de QMK). Seguidamente, usamos el código de jpe\cite{lvgl-jpe} para usar esta librería. \newline

De momento no he implementado esta librería puesto que añadiría bastante complejidad a \textbf{\ref{section:dibujar-usb} \nameref{section:dibujar-usb}}
\hr

    \subsubsection{Errores de compilación}
    Es posible que al intentar compilar obtengamos un error parecido a \newline
    \icode{error: array subscript 0 is outside array bounds of `uint16\_t[0]' [-Werror=array-bounds]}, este error se debe a un cambio en \textbf{gcc 12}. Seguramente estará solucionado en un par de meses con una actualización en QMK, pero de no ser así podemos revertir a \textbf{gcc 11} para que el mismo código se pueda compilar. Pero también podemos arreglarlo manualmente con: \newline
    \begin{multicli}
        \cliarrow sudo pacman -{}-needed -U https://archive.archlinux.org/packages/a/arm-none-eabi-gcc/arm-none-eabi-gcc-11.3.0-1-x86\_64.pkg.tar.zst
    \end{multicli}

Una vez configurado QMK, creamos los archivos básicos para el firmware de nuestro teclado haciendo \cli{qmk new-keyboard} e introduciendo los datos necesarios, sin embargo esto crea una carpeta directamente en la ruta \icode{qmk\_firmware/keyboards}, en mi caso he creado una carpeta nueva bajo este directorio con mi nick (elpekenin) como nombre, y he movido la carpeta del teclado ahí dentro, de forma que si en un futuro diseño otro teclado se guarde en esta misma carpeta.

Para acelerar el proceso de compilado podemos guardar nuestra configuración (teclado y keymap) y el número de hilos que usará el compilador
\begin{multicli}
    \cliarrow qmk config user.keyboard=elpekenin/access \newline
    \cliarrow qmk config user.keymap=default \newline
    \cliarrow qmk config compile.parallel=20 \newline
    \cliarrow qmk config flash.parallel=20
\end{multicli}

\subsection{Driver para pantalla ili9486}
Para hacer pruebas antes de diseñar y fabricar la PCB, he usado un módulo que integra una pantalla, sensor táctil y lector de tarjeta SD. \newline
Sin emabargo, QMK no tiene soporte para este modelo concreto de pantalla. Por tanto, usando como base el código\cite{ili9488} para otro dispositivo de la misma familia, y con ayuda de los usuarios @sigprof y @tzarc he desarrollado un driver\cite{ili9486-pr} para la pantalla usada. \newline
Los cambios se reducen a dos cosas:
\begin{itemize}
    \item Cambiar la fase de inicialización de la pantalla, que configura valores como el formato en el que se envían los píxeles a mostrar
    \item Puesto que la pantalla contiene un circuito que convierte la línea SPI en una señal paralela de 16 bits que se envía a la pantalla, mediante registros de desplazamiento, tendremos problemas al enviar información de un tamaño que no sea múltiplo de 16, por tanto hacemos un par de cambios al código ``normal''.
\end{itemize}

\subsection{Dibujar por USB} \label{section:dibujar-usb}
Lo siguiente que necesitamos resolver es controlar la pantalla desde el ordenador, de forma que podamos dibujar en ella desde nuestro software, para esto usaremos la funcionalidad XAP\cite{xap} (aún en desarrollo), que nos proporciona un nuevo endpoint HID\cite{hid} sobre el bus USB y un protocolo\cite{xap-specs} que podemos extender. \newline
Ahora podemos definir nuestros mensajes simplemente creando un archivo \icode{xap.hjson}\cite{xap-custom-msg} \newline
Una vez creados los mensajes, añadimos el código\cite{qp-xap-handler} que se ejecuta cuando los recibamos

\subsection{Mejora algoritmo elipses}
Durante las pruebas con el script anterior, vi que el algoritmo para dibujar elipses funcionaba regular, por lo que hice un \textit{refactor} del mismo. Es decir, mantenemos la misma ``forma'' (atributos que se reciben y valor devuelto), pero la implementación\cite{ellipse-pr} es compleamente diferente. Sigue sin ser perfecto, pero parece funcionar mejor.

\subsection{Driver para sensor XPT2046}
Al igual que hemos necesitado un driver (código que nos permite comunicarnos con el hardware) para dibujar en la pantalla, necesitamos otro para poder leer la posición en la que esta ha sido pulsada. Por desgracia, QMK no tiene ninguna característica parecida, por lo que usaremos el código\cite{touch-driver} que he escrito para ello desde 0. \par
Este hardware, al igual que la pantalla, también tiene una particularidad, y es que la función \icode{spi\_receive()} proporcionada por QMK, realmente es un intercambio de información en vez de solo lectura, esto es debido a que SPI es un protocolo síncrono. \newline 
El problema aparece porque QMK envía todos los bits a 1 -dado que algunos microcontroladores hacen esto por hardware y no se puede cambiar- para poder recibir información, \textbf{sin embargo}, el XPT2046 (a diferencia de la mayoria de chips) utiliza esta información que recibe mientras que nos reporta su información para cambiar su configuración. De forma, que en caso de que los 2 últimos bits que enviamos no tengan el valor adecuado, no podremos usar el sensor. Lo bueno es que la solución es tan simple como usar \icode{spi\_write(0)} para enviar un byte vacío y recibir la respuesta dejando una configuración conveniente.

\subsection{Generar imágenes}
Dado que vamos a usar varios iconos para hacer la interfaz gráfica en la pantalla, he cogido la colección de iconos Material Design Icons\cite{templarian-mdi} y he convertido todos sus iconos al formato que utiliza QMK para representar imágenes. Las imágenes originales, scripts usados para la conversión y los archivos en formato QGF\cite{qgf} se pueden encontrar en este repositorio\cite{mdi-qgf}

\subsection{Control con una mano}
Para dotar de accesibilidad al teclado, vamos a incorporar una funcionalidad opcional que permita que los LEDs debajo de cada tecla se queden apagados, excepto uno que será usado como ``selector''. Esta animación se define con el archivo \icode{rgb\_matrix\_kb.inc}\ref{one-hand-anim}, que utiliza una variable global para controlar la dirección en la que nos movemos. Al ser una variable global, la podremos editar en el código de usuario (lo que en QMK se conoce como \textit{keymap}), de forma que cada persona pueda añadir \textit{triggers} acorde a sus necesidades, usando joysticks, touchpads, un conjunto de unas pocas teclas, ... \par
Hay que recordar que lo que acabamos de hacer es solo una animación, es decir, solo sirve para cambiar el estado de los LED bajo cada tecla. Procedemos por tanto a modificar el código que escanea el estado de cada tecla, de forma que podamos modificar su estado acorde a otro trigger, por ejemplo, con un joystick podríamos usar las 4 direcciones para mover el selector y su pulsación para pulsar virtualmente la tecla seleccionada.

\subsection{Añadir nuestro código}
Para poder usar nuestros nuevos archivos, no es suficiente con añadir un \icode{\#include} sino que también debemos indicarle a las herramientas de compilado que ``busquen'' archivos en la carpeta donde tenemos el código, en este caso la carpeta base del teclado, asi como subcarpeta \icode{code}. También es necesario, no tengo muy claro por qué, añadir los archivos \icode{.c} de las imagenes QGF o no podremos usarlos. Este archivo se complica un poco porque no debemos incluir algunos archivos si no están algunas opciones habilitadas, por ejemplo, no debemos añadir las imágenes si no tenemos la opción de usar la pantalla habilitada. Esto lo hacemos con el archivo \icode{rules.mk}\cite{rules-mk}

\subsection{Comunicación con el software de PC}
La idea principal del control de la pantalla es que el teclado no dibuje en ella nada, de forma que la controlemos exclusivamente desde el programa desarrollado. De esta manera, lo que haremos será leer el estado del sensor táctil cada 200ms y, en caso de estar pulsado, enviaremos las coordenadas de dicha pulsación. Adicionalmente enviamos otro mensaje como si se hubiera pulsado la posicion (0, 0), donde no hay ninguna lógica, para que se ``limpie'' la pantalla al acabar una pulsación. El código relevante es: \code{access/keymaps/default}{keymap.c}
Al igual que hemos dicho antes con añadir o no añadir un archivo a nuestro código según la configuración, debemos hacer lo mismo con bloques de código, usando \icode{\#ifdef} de forma que no intentemos usar la pantalla táctil o enviar un mensaje XAP si estas características no se han activado.
