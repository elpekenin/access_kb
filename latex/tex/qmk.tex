\section{Instalación}
Para poder usar QMK tenemos que instalar su CLI (interfaz en la línea de comandos), primero instalamos 2 paquetes que nos harán falta \cli{sudo pacman -S python git} y luego instalamos la utilidad con \cli{pip install qmk} ya que se trata de una librería escrita en Python.

\section{Configuración}
Tras esto, descargamos el repositorio de QMK con \cli{qmk setup}, a este comando le podemos pasar como parámetro nuestro fork del repositorio (en mi caso \cli{qmk setup elpekenin/qmk\_firmware}), para poder usar git, ya que no tendremos permisos en el repositorio oficial. Este comando también nos instalará los compiladores necesarios y comprobará las udev de nuestro sistema Linux, para que podamos trabajar con los dipositivos sin problema, en caso de que no estén bien configuradas podemos usar copiar el archivo \icode{50-qmk.rules} en \icode{/etc/udev/rules.d/}. Finalmente podemos ejecutar \cli{qmk doctor} para comprobar el estado de QMK.
Si estamos en Linux, es posible que tengamos problemas con \textbf{udev}, las reglas que controlan los permisos sobre dispositivos. Para solventar esto copiamos el archivo \icode{50-qmk.rules} en \icode{/etc/udev/rules.d/} 
Para poder hacer debug con \cli{qmk console} he necesitado añadir una línea a dicho archivo: \newline
\icode{KERNEL==``hidraw'', SUBSYSTEM==``hidraw'', MODE=``0666'', TAG+=``uaccess'', TAG+=``udev-acl''} \vspace{0.5cm} 
Opcionalmente, podemos instalar LVGL funciones gráficas más complejas en la pantalla LCD, en vez de usar el driver de QMK. Esto lo haremos con \newline
\cli{git submodule add -b release/v8.2 https://github.com/lvgl/lvgl.git lib/lvgl} (desde el directorio base de QMK). Seguidamente, usamos el código de \mycite{lvgl} para usar esta librería. \mybreak

    \subsection{Errores de compilación}
    Es posible que al intentar compilar obtengamos un error parecido a \newline
    \icode{error: array subscript 0 is outside array bounds of `uint16\_t[0]' [-Werror=array-bounds]}, este error se debe a un cambio en \textbf{gcc 12}. Seguramente estará solucionado en un par de meses con una actualización en QMK, pero de no ser así podemos revertir a \textbf{gcc 11} para que el mismo código se pueda compilar.

Una vez configurado QMK, creamos los archivos básicos para el firmware de nuestro teclado haciendo \cli{qmk new-keyboard} e introduciendo los datos necesarios, sin embargo esto crea una carpeta directamente en la ruta \icode{qmk\_firmware/keyboards}, en mi caso he creado una carpeta nueva bajo este directorio con mi nick (elpekenin) como nombre, y he movido la carpeta del teclado ahí dentro, de forma que si en un futuro diseño otro teclado se guarde en esta misma carpeta.

\newpage
Para acelerar el proceso de compilado podemos guardar nuestra configuración (teclado, keymap y número de hilos que se usan):
\begin{multicli}
    \cliarrow qmk config user.keyboard=elpekenin/access  \\
    \cliarrow qmk config user.keymap=default \\
    \cliarrow qmk config compile.parallel=8
\end{multicli}

\vspace*{0.5cm}
