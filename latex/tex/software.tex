Para poder intercambiar información con el teclado de forma que podamos configurarlo o enviarle información en vez de simplemente escuchar las teclas que se han pulsado, vamos a desarrollar un programa en Tauri\cite{tauri} (librería escrita en Rust\cite{rust}) ya que permite usar el mismo código en multitud de sistemas operativos gracias a que funciona internamente con un servidor HTML. 

\subsection{Instalación}
Para instalar Rust podemos usar pacman \cli{sudo pacman -S rust}, sin embargo para desarrollar código es preferible usar un script que nos proporciona la comunidad del lenguaje, y que permite cambiar fácilmente la versión del lenguaje con la que compilamos, tan sólo necsitamos ejecutar \cli{curl --proto '=https' --tlsv1.3 -sSf https://sh.rustup.rs -o rust.sh}

La comunicación con el teclado se realiza usando la librería hidapi\cite{hidapi}, a la que accedemos desde Rust gracias al wrapper\cite{hidapi-rs} que implementa una ``pasarela'' a la librería en C.

Primero instalamos NodeJS\cite{nodejs} con \cli{sudo pacman -S nodejs}, después clonamos el repositorio\cite{karl-xap}, usado de base (y en el que he colaborado) para desarrollar el software. \newline 
Para instalar las dependencias de JavaScript ejecutamos \cli{npm i}. Ahora ya podemos correr \cli{cargo tauri dev} para lanzar nuestro programa. \par

\hr
He tenido que desactivar la opción wgl(libreria de Windows para OpenGL) en VcXsrv para que funcionase
\hr
