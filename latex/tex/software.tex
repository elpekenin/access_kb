Para poder intercambiar información con el teclado de forma que podamos configurarlo o enviarle información en vez de simplemente escuchar las teclas que se han pulsado, vamos a desarrollar un programa que se ejecute en el ordenador. 

Vamos a usar \mycite{tauri} ya que permite usar el mismo código en multitud de sistemas operativos, esto se consigue gracias a que funciona internamente con un servidor HTML, por lo que se puede ejecutar en diversas plataformas.

Primero instalamos tauri:
\begin{multicli}
    \textcolor{green}{\# Dependencias} \newline
    \cliarrow sudo pacman -Syu \newline
    \cliarrow sudo pacman -S --needed \textbackslash \newline
    \mytab webkit2gtk \textbackslash \newline
    \mytab base-devel \textbackslash \newline
    \mytab curl \textbackslash \newline
    \mytab wget \textbackslash \newline
    \mytab openssl \textbackslash \newline
    \mytab appmenu-gtk-module  \textbackslash \newline
    \mytab gtk3 \textbackslash \newline
    \mytab libappindicator-gtk3 \textbackslash \newline
    \mytab librsvg \textbackslash \newline
    \mytab libvips \newline
    \cliarrow curl --proto '=https' --tlsv1.2 https://sh.rustup.rs -sSf | sh \newline
    \cliarrow cargo install tauri-cli \newline

    \textcolor{green}{\# Crear proyecto} \newline
    \cliarrow npm create tauri-app
\end{multicli}

Después creamos el proyecto de \mycite{astro}, que es el framework que voy a utilizar para hacer el frontend (interfaz gráfica) de la aplicación \cli{npm create astro@latest}.Tras esto, modificamos el archivo ce configuración de Tauri para que ejecute el servidor de Astro:
\code{software/src-tauri}{tauri.conf.json}

\newpage
Asímismo tenemos que editar la configuración de Astro para que se ejecute en el mismo puerto que hemos configurado, y de paso he añadido la opción \textbf{host} para que el servidor Astro sea visible desde otros dispositivos de la red local
\code{software/frontend}{astro.config.mjs}

En este momento tenemos la siguiente estructura de archivos:
\begin{multicli}
\cliarrow tree -L 2 \newline
.  \newline
├── astro.config.mjs \newline
├── package.json \newline
├── package-lock.json \newline
├── src \newline
│   ├── components  \newline
│   ├── env.d.ts \newline
│   ├── layouts \newline
│   └── pages \newline
├── src-tauri \newline
│   ├── build.rs \newline
│   ├── Cargo.lock \newline
│   ├── Cargo.toml \newline
│   ├── icons \newline
│   ├── src \newline
│   ├── target \newline
│   └── tauri.conf.json \newline
└── tsconfig.json
\end{multicli}

Ejecutando \cli{cargo tauri dev} podremos lanzar nuestro programa. \par

\myemph{NOTA}: He tenido que desactivar la opción wgl(libreria de Windows para OpenGL) en VcXsrv para que funcionase
