Para poder intercambiar información con el teclado de forma que podamos configurarlo o enviarle información en vez de simplemente escuchar las teclas que se han pulsado, vamos a desarrollar un programa en Tauri\cite{tauri} (librería escrita en Rust\cite{rust}) ya que permite usar el mismo código en multitud de sistemas operativos gracias a que funciona internamente con un servidor HTML. 

\subsection{Instalación}
Para instalar Rust podemos usar pacman \cli{sudo pacman -S rust}, sin embargo para desarrollar código es preferible usar un script que nos proporciona la comunidad del lenguaje, y que permite cambiar fácilmente la versión del lenguaje con la que compilamos, tan sólo necsitamos ejecutar\newline
\cli{curl --proto '=https' --tlsv1.3 -sSf https://sh.rustup.rs -o rust.sh} \vspace{.1cm}\newline

La comunicación con el teclado se realiza usando la librería hidapi\cite{hidapi}, a la que accedemos desde Rust gracias al wrapper\cite{hidapi-rs} que implementa una ``pasarela'' a la librería en C. Para instalarla tan solo necesitamos añadirla al archivo \icode{Cargo.toml} de nuestro proyecto \vspace{0.1cm}\newline

Primero instalamos NodeJS\cite{nodejs} con \cli{sudo pacman -S nodejs}, después clonamos el repositorio\cite{karl-xap}, usado de base (y en el que he colaborado) para desarrollar el software. \newline 
Para instalar las dependencias de JavaScript ejecutamos \cli{npm i}. Ahora ya podemos correr \cli{cargo tauri dev} para lanzar nuestro programa. \par

\hr
He tenido que desactivar la opción wgl(libreria de Windows para OpenGL) en VcXsrv para que funcionase
\hr

\subsection{Desarrollo}
    \subsubsection{Escuchar nuestros mensajes}
    Lo primero que vamo a añadir es la capacidad de poder recibir los mensajes que nos llegan desde el teclado, para ello definimos un struct que indique el formato del mensaje, podemos ver algo de código que no es necesario entender, lo importante son los dos \icode{u16} \newline
    \rust{xap-specs/src/protocol}{broadcast.rs}

    \newpage
    Acto seguido, hacemos que el \textit{eventloop} de nuestra aplicación escuche los mensajes, ya que por defecto los ignora. Primero definimos un nuevo evento dentro del listado de tipos de evento
    \rust{src-tauri/src}{events.rs}

    Añadimos el código necesario para publicar este nuevo tipo de evento al recibir los mensajes correspondientes
    \rust{src-tauri/src/xap/hid}{device.rs}

    Y por último hacemos la relación entre este evento la lógica de usuario que se debe ejecutar con ellos.
    \rust{src-tauri/src}{main.rs}

    Toda la lógica de usuario (personalizable), se puede encontrar en el archivo \icode{src-tauri/src/user.rs}\cite{user-rs}

    \newpage
    \subsubsection{Correr aplicación en segundo plano}
    Dado que todo el contenido de la pantalla, así como su funcionalidad al pulsarla dependen de nuestra aplicación, vamos a hacer todo lo posible para que se mantenga abierta. Para ello hacemos que al pulsar el boton de cerrar, la app se minimice en vez de terminar y le añadimos un \textit{systray} para tener un icono en la barra de tareas donde podamos volver a ponerla en pantalla
    \rustcounter{src-tauri/src}{main.rs}{2}

    \newpage
    \subsubsection{Indicador de conexión}
    Vamos a añadir también un indicador en la pantalla, de forma que Tauri nos haga saber cuando se conecta o desconecta del teclado, evitando que intentemos usar la pantalla cuando no va a funcionar. Editamos el eventlooop y definimos las nuevas funciones
    \rustcounter{src-tauri/src}{main.rs}{3}
    \rust{src-tauri/src}{user.rs}

    \subsubsection{Arrancar HomeAssistant}
    Dado que nuestro programa se tiene que comunicar con la domótica de casa, vamos a asegurarnos de que esté ejecutándose, iniciándolo al abrir el programa. Al inicio del archivo \icode{main.rs} ejecutamos \icode{user::on\_init();}
    \rustcounter{src-tauri/src}{user.rs}{2}
