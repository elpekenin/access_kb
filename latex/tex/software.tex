Para poder intercambiar información con el teclado de forma que podamos configurarlo o enviarle información en vez de simplemente escuchar las teclas que se han pulsado, vamos a desarrollar un programa que se ejecute en el ordenador. 

Vamos a usar \mycite{tauri} ya que permite usar el mismo código en multitud de sistemas operativos, esto se consigue gracias a que funciona internamente con un servidor HTML, por lo que se puede ejecutar en diversas plataformas.

Primero instalamos tauri:
\begin{multicli}
    \textcolor{green}{\# Dependencias} \\
    \cliarrow sudo pacman -Syu \\
    \cliarrow sudo pacman -S --needed \textbackslash \\
    \mytab webkit2gtk \textbackslash \\
    \mytab base-devel \textbackslash \\
    \mytab curl \textbackslash \\
    \mytab wget \textbackslash \\
    \mytab openssl \textbackslash \\
    \mytab appmenu-gtk-module  \textbackslash \\
    \mytab gtk3 \textbackslash \\
    \mytab libappindicator-gtk3 \textbackslash \\
    \mytab librsvg \textbackslash \\
    \mytab libvips \\
    \cliarrow curl --proto '=https' --tlsv1.2 https://sh.rustup.rs -sSf | sh \\
    \cliarrow cargo install tauri-cli \\

    \textcolor{green}{\# Crear proyecto} \\
    \cliarrow npm create tauri-app
\end{multicli}

Después creamos el proyecto de \mycite{astro}, que es el framework que voy a utilizar para hacer el frontend (interfaz gráfica) de la aplicación \cli{npm create astro@latest}.Tras esto, modificamos el archivo ce configuración de Tauri para que ejecute el servidor de Astro: \vspace{1cm} \newline
\file{software/src-tauri}{tauri.conf.json}{Cambios necesarios en Tauri}

\newpage
Asímismo tenemos que editar la configuración de Astro para que se ejecute en el mismo puerto que hemos configurado, y de paso he añadido la opción \textbf{host} para que el servidor Astro sea visible desde otros dispositivos de mi red local, puesto que el TFM lo estoy desarrolando vía SSH en otro ordenador con Linux
\file{software/frontend}{astro.config.mjs}{Configuración de Astro}

En este momento tenemos la siguiente estructura de archivos:
\begin{multicli}
    \cliarrow tree -L 2 \\
    . \\
    ├── frontend \\
    │   ├── astro.config.mjs \\
    │   ├── node\_modules \\
    │   ├── package.json \\
    │   ├── package-lock.json \\
    │   ├── public \\
    │   ├── README.md \\
    │   └── tsconfig.json \\
    ├── README.md \\
    └── src-tauri \\
    \hphantom{0.1cm}├── build.rs \\
    \hphantom{0.1cm}├── Cargo.lock \\
    \hphantom{0.1cm}├── Cargo.toml \\
    \hphantom{0.1cm}├── icons \\
    \hphantom{0.1cm}├── target \\
    \hphantom{0.1cm}└── tauri.conf.json
\end{multicli}
