\section{Configuración inicial}
Para poder intercambiar información con el teclado de forma que podamos configurarlo o enviarle información en vez de simplemente escuchar las teclas que se han pulsado, vamos a desarrollar un programa que se ejecute en el ordenador. 

Usaremos \mycite{tauri} ya que permite usar el mismo código en multitud de sistemas operativos, esto se consigue gracias a que funciona internamente con un servidor HTML, por lo que se puede ejecutar en diversas plataformas.

Instalamos \mycite{nodejs} con \cli{sudo pacman -S nodejs} para poder usar \mycite{vue}, que es el framework con el que haremos el frontend (interfaz gráfica) de la aplicación y clonamos el repositorio de \mycite{karl-xap}, que he usado de base (y al que he colaborado) para desarrollar el software. Para instalar las dependencias de JavaScript ejecutamos \cli{npm i}. Ahora ya podemos correr \cli{cargo tauri dev} para lanzar nuestro programa. \par

\hr
He tenido que desactivar la opción wgl(libreria de Windows para OpenGL) en VcXsrv para que funcionase
\hr
